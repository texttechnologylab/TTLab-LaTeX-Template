% arara: lualatex: { shell: yes }
% arara: biber
% arara: nomencl
% arara: lualatex: { shell: yes }
% arara: lualatex: { shell: yes }

\documentclass[
  %%% Optional Flags:
  minted, % for code type setting
  % mintedframed, % add simple frames around all minted environments
  tcolorbox, % colorful boxes and callouts
  % algorithm, % for algorithms
  % nomencl, % for nomenclatures
  % nolibertine, % switch off Libertine Fonts
  %%% REQUIRED: select your Studiengang!
  % bsc2019,
  % msc2019,
  % mscbio2019,
  % mscwirtschaft2019,
  % seminar,
  %%% Language Options: choose one!
  % ngerman
  english
]{ttlab-qualify}

% Add your references to this file.
% Good sources are DBLP and Google Scholar.
% Make sure to double check any references to arXiv or other pre-print servers,
% as there may be a more up-to-date version published at a conference later!
\addbibresource{references.bib}

%% only needed for the template %%
\usepackage{lipsum}

% Fill in your personal data
\titlehead{
  Name\\
  Matrikelnummer\\
  Studiengang (BA / MA)\\
  Studienfachkombination / Schwerpunkt \\
  Semester\\
  E-Mail-Adresse  
}
\subject{Titel der Veranstaltung/Art der Arbeit (BA/MA)}
\author{Name des Verfassers}
\title{Titel der Arbeit}
\subtitle{Ggf. Untertitel}
\date{Abgabedatum: <Datum>}
\publishers{Name des Instituts\\Name des Betreuers\\Ggf. Name des Zweitbetreuers}

\begin{document}
{
    \hypersetup{hidelinks}
    \maketitle

    %%% Writing the thesis in English requires an additional German abstract!
    \begin{zusammenfassung}
        ...
    \end{zusammenfassung}
    
    \begin{abstract}
        ...
    \end{abstract}
    
    \tableofcontents
    \listoffigures
    \listoftables
    \listoflistings
    % \listof??? % other lists if necessary
}

\chapter{Template}
    \section{Tables}
        Create tables using \href{https://ctan.org/pkg/tabularray?lang=en}{tabularray} in the \texttt{tblr} environment.
        %
        It supersedes all other table packages like \texttt{tabluarx}.
        
        \begin{table}[hb]
            \centering
            \caption{Captions \textit{above} tables.}\label{tab:example}
            \begin{tblr}{colspec={llll}}
                \toprule
                Alpha   & Beta  & Gamma   & Delta \\
                \midrule
                Epsilon & Zeta  & Eta     & Theta \\
                \cmidrule{1-3}
                Iota    & Kappa & Lambda  & Mu    \\
                \cmidrule{2-4}
                Nu      & Xi    & Omicron & Pi    \\
                \bottomrule
            \end{tblr}
        \end{table}
    
    \section{Figures}
        \begin{figure}[hb]
            \centering
            \includegraphics[width=0.4\textwidth]{figures/logos/logo-gu-blau.pdf}
            \caption{Captions \textit{below} figures.}\label{fig:example}
        \end{figure}
        
        \clearpage
    \section{Code Typesetting}
        Use \href{https://ctan.org/pkg/minted}{minted} to typeset code snippets.
        
        \begin{listing}[hb]
            \captionabove{Example listing of C-code.}\label{listing:example}
            \begin{minted}[
                frame=lines,
                framesep=2mm,
            ]{C}
                int main() {
                    printf("hello, world");
                    return 0;
                }
            \end{minted}
        \end{listing}


        % Define an environment for Python code
        \newtcblisting{tcbPython}[1][]{
            % the colors are only for demonstration, not required
            colback=blue!5!white,
            colframe=blue!75!black,
            listing only,
            left=5mm, 
            enhanced,
            overlay={\begin{tcbclipinterior}\fill[red!20!blue!20!white] (frame.south west)
            rectangle ([xshift=5mm]frame.north west);\end{tcbclipinterior}},
            minted language=python3,
            minted options={numbersep=3mm},
            #1
        }       
        \begin{listing}[hb]
            \captionabove{Fancy Python typesetting in a \href{https://ctan.org/pkg/tcolorbox?lang=en}{tcolorbox}.}\label{listing:example}
            \begin{tcbPython}
if __name__ == "__main__":
    print("Hello World!")
            \end{tcbPython}
        \end{listing}

    
    \section{References}
        Add references by updating the \textit{references.bib} file with appropriate Bib\TeX entries and insert citations like this \citep{lastname-etal-2025-example}.
        %
        Check out the \href{https://tug.ctan.org/info/biblatex-cheatsheet/biblatex-cheatsheet.pdf}{Bib\LaTeX{} Cheatsheet}!
        
        \begin{table}[h]
            \centering
            \caption{Example citation commands and their output.}
            \begin{tblr}{
                    colspec={lX},
                    row{1}={font=\bfseries}
                }
                \toprule
                Commands & Output                                                                      \\
                \midrule
                {\fakeverb{\textcite}  \\ \fakeverb{\citet}} & \citet{lastname-etal-2025-example}      \\
                {\fakeverb{\parencite} \\ \fakeverb{\citep}} & \citep{lastname-etal-2025-example}      \\
                \fakeverb{\citealp}                          & \citealp{lastname-etal-2025-example}    \\
                \fakeverb{\citeauthor}                       & \citeauthor{lastname-etal-2025-example} \\
                \fakeverb{\fullcite}                         & \fullcite{lastname-etal-2025-example}   \\
                \bottomrule
            \end{tblr}
        \end{table}

\chapter{Structure \& Layout}
    \lipsum[1]
    
    \section{Section}
        \lipsum[2]
        
        \subsection{Subsection}
            \lipsum[3]
            
            \subsubsection{Subsubsection}
                \lipsum[4]
                
                \minisec{Minisec}
                    \lipsum[5]
                
                \paragraph{Paragraph}
                    \lipsum[6]
                    
% Bibliography
\nocite{lastname-etal-2025-example}
\printbibliography

\appendix
\chapter{First Appendix}

\end{document}
